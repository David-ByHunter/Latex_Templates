\documentclass[12pt]{article}
% List of packages
\usepackage[a4paper, margin = 1.5cm]{geometry} % package for changing the layout
\setlength{\parindent}{0pt} % Indent for each paragraph
\setlength{\parskip}{0.4em}    % Extra space between paragraphs
\usepackage[table]{xcolor}
\usepackage{amsmath} % Package for maths
\usepackage{amssymb} % Package for symbols
\usepackage{derivative}
\usepackage{physics}
\usepackage{steinmetz} % For using phasor in eqn
\usepackage{float} % Para poder usar H en las figure
\usepackage{graphicx} % Package for inluding graphic
\usepackage{caption} % Package for using \ContinuedFloat
\usepackage{subcaption} % Package for using subfigure
\usepackage{setspace} % Package for editing the table of contents
\usepackage{hyperref} % Package for creating hyèrlinks
\usepackage{tikz} % Package for creating graphics
\usepackage[siunitx]{circuitikz} % Package for creating circuits
\usetikzlibrary{shapes.geometric, arrows} % Libraries for tikz to create flowcharts
\usetikzlibrary{mindmap} % Libraries for tikz to create mindmaps
\usepackage{listings} % Package for highlight text
\usepackage{color}
\usepackage{pdfpages}
\usepackage[outdir=./]{epstopdf}
% Package for tables
\usepackage{siunitx} % Required for alignment of tables
\usepackage{multirow} % Required for multirows
\usepackage{booktabs} % For prettier tables
\usepackage{longtable} % To display tables on several pages
\usepackage{rotating} % To display tables in landscape
\usepackage{pgfplotstable} % Generates table from .csv
\pgfplotsset{compat=newest} % Allows to place the legend below plot
\usepgfplotslibrary{units} % Allows to enter the units nicely
% Font Set up
\usepackage[default]{opensans}
% \usepackage[defaultsans]{opensans} % To set Open Sans as default sans-serif only
% \usepackage[sfdefault]{roboto}  %% Option 'sfdefault' only if the base font of the document is to be sans serif
\usepackage[T1]{fontenc}
% Set up of packages
\sisetup{
	round-mode          = places, % Rounds numbers
	round-precision     = 2, % to 2 places
}
% Footer
\usepackage{fancyhdr}
\pagestyle{fancy}
\fancyhf{}
\renewcommand{\headrulewidth}{0pt}
%\pagestyle{plain}
\rfoot{Realizado por \href{https://davidbarfer.com}{David Barrero} \hspace{0.5cm} \thepage}
% Change Language of things
\renewcommand{\figurename}{Figura}
\renewcommand{\tablename}{Tabla}
\renewcommand{\contentsname}{Tabla de Contenido}
\renewcommand{\listfigurename}{Lista de figuras}
\renewcommand{\listtablename}{Lista de tablas}
% Document
\begin{document}
	\begin{titlepage}
		\pagenumbering{gobble}		
	%%	\includepdf[pages=-]{Portada.pdf} % Insert a pdf as cover
		\raggedleft % Right align the title page
	
		\rule{1pt}{\textheight} % Vertical line
		\hspace{0.05\textwidth} % Whitespace between the vertical line and title page text
		\parbox[b]{0.75\textwidth}{ % Paragraph box for holding the title page text, adjust the width to move the title page left or right on the page
		
		{\Huge\bfseries Title\\[0.5\baselineskip] Subtitle\\[0.2\baselineskip] Date}\\[2\baselineskip] % Title
		{\Large{Name Surname \\[0.5\baselineskip] Name Surname}} % Author name
		
		\vspace{0.5\textheight} % Whitespace between the title block and the publisher	
		}
	\end{titlepage}
	\newpage
	\pagenumbering{arabic}
	\tableofcontents
	\newpage
	\section{Tables}
	\begin{table}[H]
		\centering
	  \rowcolors{2}{gray!25}{white}
		\begin{tabular}{|c|c|}
		  \rowcolor{gray!50}
		  \hline
			Head 1 & Head 2 \\
			\hline
			Example 1 & Example 2 \\
			Example 3 & Example 4 \\
			Example 5 & Example 6 \\
			\hline
		\end{tabular}
		\caption{This is an example of a table}
		\label{tab: 01}
	\end{table}
	\section{Apéndice}
	\listoffigures
	\listoftables
\end{document}